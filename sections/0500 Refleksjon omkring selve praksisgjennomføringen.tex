\section{Refleksjon omkring selve praksisgjennomføringen}

% I hvilken grad har man nådd sine læringsmål, jf. praksisplan?
\subsection{Læringsmål og måloppnåelse}
De oppsatte læringsmålene ble i stor grad oppnådd:
\begin{itemize} 
    \item \textbf{God praksis i programvareutviklings:} Jeg fikk praktisk erfaring med verktøy som \Gls{git} og Postman, samt systematisk testing av \gls{api}-er. Det var imidlertid rom for forbedring, spesielt i konsekvent bruk av systemutviklingsmetodikker som \gls{issue_board}. 
    \item \textbf{Forskning for implementering av funksjoner:} Flere oppgaver krevde egenstudier, særlig design av \gls{sql}-queryer og integrasjon mellom backend og frontend. Jeg lærte å finne og tilpasse beste praksis til våre behov. 
    \item \textbf{Forbedring av språkferdigheter:} Jeg utviklet betydelig forståelse i Java, Python, \gls{sql}, og Vue3. Python ble brukt minimalt, men \gls{sql}-ferdighetene ble forbedret. 
    \item \textbf{Domene-spesifikk kunnskap:} Arbeidet med \gls{trace} ga innsikt i hvordan logistikkdata brukes til analyse og kostnadsfordeling. 
\end{itemize}

% Ble praksisen som forventet?
\subsection{Forventninger og faktisk gjennomføring}
Praksisen møtte forventningene i teknisk innhold og veiledning, men det ble avvik fra planen når det gjelder metodisk struktur og tidsplan. Eksempelvis var bruken av \gls{issue_board} begrenset, og jeg måtte komprimere arbeidsmengden mot slutten av perioden.

% Hvilke erfaringer har man gjort?
% Avvik mellom plan og faktisk praksisgjennomføring?
\subsection{Erfaringer og avvik fra planen}
Jeg erfarte viktigheten av fleksibilitet og selvstendighet, særlig når veileder ikke alltid var tilgjengelig. Balansen mellom praksis og studier ga verdifull innsikt i tidsstyring.
Avvik fra planen inkluderte:
\begin{itemize} 
    \item \textbf{Arbeidstid:} Arbeidet ble intensivert mot slutten av perioden. 
    \item \textbf{Metodikk:} Systemutviklingsverktøy og metodikker ble ikke konsekvent brukt, og møtene ble mindre hyppige enn planlagt. 
\end{itemize}

% Hvordan var oppfølgingen?
% Hadde man de nødvendige forutsetninger for å løse oppgavene?
\subsection{Oppfølging og forutsetninger for oppgavene}
Veilederen Andrea Tenti ga tilstrekkelig støtte, men begrenset tilgjengelighet gjorde at jeg måtte søke hjelp fra andre ansatte, noe som fremmet selvstendig problemløsning. Jeg hadde nødvendige forutsetninger fra studiet for å kunne løse oppgavene, men praksisen krevde også læring av nye teknologier, som Vue3 og avanserte \gls{sql}-queryer.

% Hva har man lært?
% Var oppgavene relevant i forhold til egen utdanning?
% Var utført arbeid nyttig og relevant for virksomheten?
% Erfaringer og refleksjoner som kobler praksisen og det teoretiske stoffet i utdanningen.
\subsection{Lærdom og relevans for utdanning og virksomhet}
Gjennom praksisen har jeg lært:

\begin{itemize} 
    \item Utvikling og testing av \gls{api}-er. 
    \item Effektiv bruk av \gls{sql} i database-integrasjoner. 
    \item Internasjonalisering i frontend-løsninger. 
    \item Samarbeid i et team med ulike roller. 
\end{itemize}

Oppgavene var svært relevante for utdanningen, og koblet teori fra algoritmer og databaser til praktisk anvendelse. Arbeidet ga virksomheten en \textit{proof of concept} for en ny modul, selv om implementering av reelle data og videre utvikling gjenstår.

% Faglig refleksjon omkring det utførte arbeidet.
% Oppnådde man de ønskede resultater, evt. hvorfor ikke?
% Refleksjon omkring fremgangsmåte og rutiner som ble fulgt.
% Hva kunne evt. vært gjort annerledes?
\subsection{Faglig refleksjon og forbedringsområder}
Arbeidet utfordret meg teknisk og bidro til å bygge ferdigheter som vil være verdifulle i fremtidige jobber. Samtidig kunne jeg forbedret timeplanlegging og systematisk bruk av utviklingsmetodikk. 
Eksempler på forbedringer inkluderer:
\begin{itemize} 
    \item Bedre planlegging for å unngå intensiv jobbing mot slutten. 
    \item Systematisk oppfølging av oppgaver i verktøy som \gls{issue_board}. 
    \item Tydeligere kommunikasjon om behov for veiledning og ressurser. 
\end{itemize}

% Utestående oppgaver?
\subsection{Utestående oppgaver og videre arbeid}
Noen oppgaver gjenstår, som implementering av reelle data, optimalisering, og brukertesting av modulen. Dette arbeidet vil bli videreført av selskapets utviklere.