\section{Oppsumering og konklusjon}

\subsection{Oppsumering av praksisperioden}

Praksisperioden hos Solwr har vært en lærerik og verdifull opplevelse som har gitt meg mulighet til å jobbe med virkelige prosjekter innenfor programvareutvikling. Hovedfokuset har vært på utvikling av en modul for \gls{trace}, hvor både backend i Java og frontend i Vue3 ble implementert. Arbeidet inkluderte også bruk av \gls{sql} for å hente og manipulere data fra databasen, samt oversettelsearbeid for å støtte internasjonalisering. \\

Gjennom praksisen ble det lagt vekt på gode utviklingsmetoder, samarbeid med andre praktikanter og veileder, samt evnen til å finne løsninger på tekniske utfordringer. Jeg oppfylte målet om å levere en fungerende \textit{proof of concept} for virksomheten.

\subsection{Læringsutbytte og refleksjoner}

Jeg har oppnådd flere viktige læingsmål:
\begin{itemize}
    \item Forbedret kunnskap i teknologier som Java, \gls{sql} og Vue.
    \item Økt forståelse for profesjonell programvareutvikling, inkludert bruk av verktøy som \Gls{git} og Postman.
    \item Erfaring med å jobbe i team og å bruke metoder for å utvikle og forbedre løsninger.
\end{itemize}

Videre har praksisen gitt verdifull innsikt i logistikkfeltet. Gjennom å jobbe med komplekse systemer, har jeg også fått styrket mine problemløsningsevner og evnene til å jobbe selvstendig. 

\subsection{Konklusjon}

Praksisen har vært en suksess både for meg og for virksomheten. Arbeidet som ble utført gir et godt utgangspunkt for videre utvikling av \gls{abc}-modulen, og det er tydelig at virksomheten ser verdien i bidraget fra praktikantene. \\

For meg har praksisen vært en viktig framvisning av sammenheng mellom teori og praksis, og den har gitt meg en bedre forståelse av hvordan kunnskapen fra utdanningen kan anvendes i et profesjonelt miljø. Dette bekrefter relevansen og verdien av å inkludere praksis i studieløpet.

\subsection{Forbedringer}

Til slutt kan det nevnes at bedre planlegging og strukturering av oppgavene, spesielt knyttet til bruk av systemutviklingsmetodikker som \gls{issue_board} og \gls{sprint}-planlegging, kunne ha gjort arbeidsprosessen mer effektiv. Til tross for dette var praksisen en positiv opplevelse som ga viktige erfaringer og nyttig læring.