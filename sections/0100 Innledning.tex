\section{Innledning}

I perioden \textbf{12. september} til \textbf{7. november 2024} gjennomførte jeg, Vegard Arnesen Mytting, praksis hos virksomheten \textbf{Solwr Software}. Praksisen fant sted ved bedriftens kontor i Ålesund, lokalisert i \href{https://www.google.com/maps/place/Solwr/@62.4774157,6.229681,17z/data=!3m1!4b1!4m6!3m5!1s0x4616da44c5c0db81:0xa0844570521977a4!8m2!3d62.4774157!4d6.229681!16s%2Fg%2F11cn6pj_2s?entry=ttu&g_ep=EgoyMDI0MTEwNS4wIKXMDSoASAFQAw%3D%3D}{Østre Gangstøvika 9, 6009 Ålesund}. Arbeidet mitt fokuserte på å utvikle en matematisk algoritme basert på regnskapsmetoden \gls{abc}, som skulle integreres som en regnskapsmodul i Solwrs web-løsning. \\

\subsection{Hensikt}
\textit{Hensikten med praksisen er å skaffe seg kunnskap om hvordan en ingeniørbedrift organiseres, ha praktisk kunnskap om ulike arbeidsteknikker og produksjonstekniske hjelpemidler, og ha innsikt i business modellen og konkurransefortrinn av en bedrift.} \cite{idata2505}\\

\subsection{Hvorfor virksomheten ønsket disse arbeidsoppgavene}
Solwr Software hadde behov for praktikanter til å gjennomføre ulike oppgaver innen matematikk, backend, og frontend. Jeg valgte det matematikk relaterte prosjektet, fordi jeg ønsker å lære mer om å implementere matematiske algoritmer i applikasjoner. Oppgaven jeg fikk, handlet om å implementere \gls{abc}-metoden, en \textit{regnskapsmodul som tilordner spesifikke kostnader til ulike produkter som selskapet produserer eller behandler.} \cite{agicap_abc} 
Denne oppgaven var relevant for virksomheten, fordi de ønsket å gi kundene sine en mulighet til å gjøre kostnadsrelevante analyser av \gls{item}ne sine. \\

\subsection{Forventninger}
Forventningene mine var å få et \textit{innblikk og erfaring fra en ingeniørarbeidsplass, for å kunne vite hva som er i ventet når jeg er ferdig utdannet. Jeg forventer at praksisens omfang skal minst være 120 timer, og i nevnt timer forventes det å få oppfølging av både virksomheten og \gls{ntnu}.} \cite{idata2505} \\

\subsection{Organisering av praksisopplegget}
Praksisopplegget hadde stor fleksibilitet. Min hovedveileder, Andrea Tenti, var tilgjengelig på tirsdager og torsdager. For de dagene han ikke var tilstedet, kunne jeg spørre Halvard Øverlien om hjelp. På grunn av dette og undervisnings-timeplanen til \gls{ntnu}, ble det tilrettelagt slik at jeg hadde hovedsakelig faste dager på onsdager og torsdager, mens andre praktikanter hadde andre dager. For å sikre god fremdrift hadde vi i \gls{rnd} avdelingen ukentlige møter hver torsdag, hvor vi diskuterte status og utfordringer i prosjektet.